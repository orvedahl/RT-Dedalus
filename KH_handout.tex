%
%     hw9.tex
%     Benjamin Brown (bpbrown@colorado.edu)
%     October 29, 2014
%
%     Problemset 9 for ASTR/ATOC 5540, Mathematical Methods, taught at
%     University of Colorado, Boulder, Fall 2014.
%

\documentclass[12pt, preprint]{aastex}
% formatting based on 2014 NASA ATP proposal with Jeff Oishi

%%%%%%begin preamble
\usepackage[hmargin=1in, vmargin=1in]{geometry} % Margins
\usepackage{hyperref}
\usepackage{url}
\usepackage{times}
\usepackage{natbib}
\usepackage{graphicx}
\usepackage{amsmath}
\usepackage{amsfonts}
\usepackage{amssymb}
\usepackage{pdfpages}
\usepackage{import}
\usepackage{listings}

\hypersetup{
     colorlinks   = true,
     citecolor     = gray,
     urlcolor       = blue
}

%% headers
\usepackage{fancyhdr}
\pagestyle{fancy}
\lhead{ASTR/ATOC 5410}
\chead{}
\rhead{22 Nov 2014}
%\rhead{SOLUTIONS}
\lfoot{Dedalus: KH}
\cfoot{\thepage}
\rfoot{Fall 2014}
% no hline under header
\renewcommand{\headrulewidth}{0pt}

\newcommand{\sol}{\ensuremath{\odot}}
\newcommand{\dedalus}{\href{http://dedalus-project.org}{Dedalus}}
\renewcommand{\vec}{\boldsymbol}
\newcommand{\del}{\vec{\nabla}}
% make lists compact
\usepackage{enumitem}
%\setlist{nosep}

%%%%%%end preamble

\begin{document}

\section*{Dedalus}
\begin{description}
     \item[Kelvin-Helmholtz with Dedalus (a 2-D PDE)]~\\ 
    Now that you have some experience with PDEs and \dedalus, let's
    take a look at the classic Kelvin-Helmholtz instability.  This one
    can be solved on your laptops.
    \begin{enumerate}
      \item Use Mercurial and clone the Kelvin-Helmholtz repository located at
       \url{https://bitbucket.org/bpbrown/kelvin_helmholtz}
       If you have Mercurial on your system, this can be done with
{\footnotesize
\begin{verbatim}
$ hg clone https://bitbucket.org/bpbrown/kelvin_helmholtz
\end{verbatim}
}
       The Kelvin-Helmholtz problem will be downloaded and
       stored in a subdirectory named \verb+kelvin_helmholtz+.
       If the \verb+hg+ command isn't recognized, please install Mercurial (\url{http://mercurial.selenic.com}).
       
       Run the basic problem by executing:
\small{
\begin{verbatim}
python3 KH_incompressible.py
\end{verbatim}
}
and plot the results with
\small{
\begin{verbatim}
python3 plot_results_parallel.py KH_incompressible slices 1 1 10
\end{verbatim}
}

       \item The equations for this Kelvin-Helmholtz problem are:
         \begin{eqnarray}
           \frac{\partial \vec{u}}{\partial t} +
             +\del P - \frac{1}{Re}\del^2 \vec{u} &=& -\vec{u}\cdot\del\vec{u}
           \\
           \frac{\partial T}{\partial t} - \frac{1}{Pe}\del^2 T&=&
           -\vec{u}\cdot\del T \\
           \vec{\del}\cdot\vec{u} &=& 0 
         \end{eqnarray}
         with vector velocity $\vec{u} = u\vec{\hat{x}} + w
         \vec{\hat{z}}$ and passive scalar tracer (here temperature)
         $T$.  Here we have non-dimensionalized the equations based on
         the large-scale flow crossing time $\tau = L/U_0$, where $L$
         is the domain size and $U_0$ is the jump in amplitude of the
         horizontal shear flow across the tanh profile.
         %
         The control
         parameters are the thermal Reynolds number $Re$ and the
         Prandtl number $Pr$, with the Peclet number $Pe=Re Pr$.  
         The initial horizontal flow and temperature field are given by a tanh.
         Open \verb+equations.py+ to find the equations
         as implemented. Note that here the boundary conditions are
         set up to continue enforcing the shear flow, and are set to
         be consistent with our initial $u$ and $T$.

         \item The provided case runs at $Re=2500$ and $Pr=1$.  If you
           increase $Re$, you may also need to increase the $nx$ and
           $nz$ resolution.  We suggest you take our approach, and
           specify the resolution in modes, then pad by a factor of
           $3/2$ for dealiasing.


         \item The initial conditions, parameters, and simulation
           resolution are all set in \verb+KH_incompressible.py+.  This is also where
           the output and analysis is done, though the actual image
           generation is done in \verb+plot_results_parallel.py+.  The
           \verb+KH.py+ file includes a somewhat similar case, but
           with a different non-dimensionalization, while
           \verb+KH_periodic.py+ applies doubly-periodic boundary conditions.
           Best of luck!

\end{enumerate}
\end{description}

\end{document}
